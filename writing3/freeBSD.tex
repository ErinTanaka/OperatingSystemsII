\section{FreeBSD}
\subsection{Virtual-Memory System}
The FreeBSD virtual memory system is based on the Mach 2.0 virtual memory system\cite{Freebsdtext}. FreeBSD supports the \lstinline{mmap} system call. This means that the address space is used in a less structured manner. For example, shared libraries might place data arbitrarily.Both the kernel and user processes use the same datastructures for managing their virtual memory systems.
\begin{itemize}
    \item \lstinline{vmspace} is used to encompass machine dependent and independent structures describing a process's address space \cite{Freebsdtext}.
    \item \lstinline{vm_map} describes the machine independent virtual address space \cite{Freebsdtext}.
    \item \lstinline{vm_map_entry} describes "... the mapping from a virtually contiguous range of addresses that share protection and inheritance attributes to the backing-store \lstinline{vm_object}"\cite{Freebsdtext}
    \item \lstinline{vm_object} is a structure used to describe the sources of data \cite{Freebsdtext}.
    \item \lstinline{shadow vm_object} is a special type of \lstinline{vm_object} that represents a modified copy of an original piece of data \cite{Freebsdtext}.
    \item \lstinline{vm_page} is the structure that represents the physical memory being used by the virtual memory system.
\end{itemize}

\subsection{Kernel Memory Management}
FreeBSD maps memory differently depending on the architecture. For a 64-bit address space architecture, memory is permanently be mapped into the high part of every process address space \cite{Freebsdtext}. In 32-bit architectures the kernel will map like the 64-bit architecture or it will switch between "...having the kernel occupy the whole address space and mapping the currently running process into the address space" \cite{Freebsdtext}. The first option for the 32-bit architecture is the most common. Switching processes will not affect the kernel portion of the address space if memory is permanently mapped to the higher part of the address space\cite{Freebsdtext}. This method also allows the kernel to freely read and write to the address space of the user process and restricts the user process from reading and writing to kernel processes\cite{Freebsdtext}.

\subsection{Slab Allocator}
In FreeBSD a slab is "...a collection of items of identical size" \cite{Freebsdtext}. Each slab is limited to the size of a single page unless the object is larger than a page in FreeBSD 11 \cite{Freebsdtext}. The single page limit prevents fragmentation. 

\subsection{Kernel Malloc}
\lstinline{Malloc()} is the preferred method of allocating kernel memory. Its interface is similar to \lstinline{malloc()} in C. \lstinline{Malloc()} takes in a parameter of the size of memory that needs to be allocated. The kernel memory allocator uses a power of 2 strategy for small allocations and switches to a large block algorithm for allocations larger than a page.

\subsection{Shared Memory}
In FreeBSD processes create shared memory with
\begin{lstlisting}
void *addr = mmap(
    void *addr, /* base address */
    size_t len, /* length of region */
    int prot, /* protection of region */
    int flags, /* mapping flags */
    int fd, /* file to map */
    off_t offset); /* offset to begin mapping */
\end{lstlisting}
This shared mapping allows processes to write to a file and those changes are able to be seen by other processes. When processes do  map the same file into their address space it is important that they are viewing the same set of pages. Files being currently being used by a lcient of the virtual memory system are represented by a \lstinline{vm_object}\cite{Freebsdtext}. If a page fault occurs, the \lstinline{vm_map_entry} structure that describes the mapping to the file is used to find the appropriate page. 

\subsection{Hardware Memory Management}
The FreeBSD memory management unit "...implements address translation and access control when virtual memory is mapped onto physical memory."\cite{Freebsdtext} There are a few different memory management unit designs that are common. The first is memory resident forward mapped page tables where the tables are large contiguous arrays \cite{Freebsdtext}. The arrays are indexed by virtual addresses. The second is inverted page table also known as reverse mapped-table. The final design is is just a translation looka-side buffer. The simple hardware design gives software flexibility.
