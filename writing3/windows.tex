\section{Windows}
\subsection{Memory Manager}
The Windows memory manager has two main tasks. The first task of the memory manager is to translate or map a process's "...virtual address space into physical memory so that when a thread running a thread running in the context of that process reads or writes to the virtual address space, the correct physical address is referenced" \cite{WindowsText}. The second job of the memory manager is "Paging some of the contents of memory to disk when it becomes overcommitted... and bringing the contents back into physical memory when needed"\cite{WindowsText}. In addition to mapping virtual address space to physical and pageing and retrieving data, the memory manager provides a set of services. These services are memory mapped files, copy-on-write memory, support for applications that use a large, sparse address spaces, and provides a way for processes to allocate large amounts of physical memory\cite{WindowsText}. The memory manager is fully re-entrantand supports
simultaneous execution on multiprocessor systems\cite{WindowsText}. This means that if two threads are running simultaneously, they can acquire the resources they need in a manner that they will not corrupt each other's data.


\subsection{Pages}
Pages are used to divide up the virtual address space. The processors that Windows runs on support two page sizes. The actual sizes vary by processor architecture but they are called large and small\cite{WindowsText}. Large pages speed up address translations but attempts to allocate large pages may fail if the system has been running for too long. Pages in a process can either be free, reserved, committed, or shareable. Committed and shareable pages become valid pages in physical memory when they are accessed\cite{WindowsText}. Committed pages cannot be shared with other processes. Shareable ones, like their label implies, can be shared with other processes but could also only be used by one.

\subsection{Shared Memory and Mapped Files}
Shared memory is implemented using \lstinline{section objects}, which are exposed as \lstinline{file mapping objects} from the memory manager \cite{WindowsText}. The \lstinline{file mapping objects} are used to map virtual addresses. A section can be accessed by one process or many. Mapped files can by used by applications to perform I/O operations by making the file appear in the application's address space\cite{WindowsText}.
