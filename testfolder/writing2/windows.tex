\section{Windows}
\subsection{I/O Manager}
The I/O manager is the main component of the Windows I/O system. It manages the communication between an I/O request packet (IRP) and the corresponding device driver \cite{WindowsText}. Majority of I/O requests are represented as IRPs in Windows. The manager creates IRPs  in memory and passes a pointer to that IRP to the driver, then it gets rid of the IRP when the necessary operation is is complete \cite{WindowsText}. The use of the I/O manager as a middleman for creation and removal of the I/O packets means that there is less work for each individual driver to do. This means that drivers are simpler and thus take up less space. 

\subsection{I/O Request Packets (IRP)}
I/O request packets are used to store information necessary to process an I/O request. Since it holds all the information that the driver needs to handle I/O requests, the structure is somewhat self-contained. Additionally, IRPs hold data that is common to all drivers in the stack and information that is unique to its specific driver\cite{WindowsText}.

\subsection{Scheduler}
The I/O manager in Windows supports five priority categories: critical, high, normal, low, and very low with a default of normal\cite{WindowsText}. The five priority levels are divided into two different prioritization modes. These modes are called strategies. The first strategy is hierarchy prioritization. Each priority has a queue and the strategy decides the order that each of the operations within those queues is processed. Hierarchy prioritization gives more important I/O requests priority over background tasks like indexing or virus scanning. The idle prioritization strategy implements a separate queue for non-idle I/O. Any non-idle, hierarchy operations get processed before the the idle I/O\cite{WindowsText}. This means that it would be possible to starve an idle I/O operation if there is one non-idle request, it would be attended to before any idle ones. The idle strategy has a timer to prevent this issue. The timer monitors the queue and ensures that at least one I/O request gets removed from the queue every half second\cite{WindowsText}. Additionally, the strategy waits for 50 milliseconds after a non-idle I/O finishes\cite{WindowsText}. This waiting period prevents any idle I/O currently on the queue from occurring in the middle of non-idle I/O.  

% \subsection{Filesystems}
% Windows supports the following filesystem formats: CDFS, UDF,FAT12, FAT16, FAT32, exFAT, and NTFS\cite{LinuxTextbook}. NTFS is the primary file system for Windows. This file system provides many features like encryption, disk quotas and security descriptors. 
